\chapter{Einführung}
\section{Einleitung}\label{sec:einleitung}
Neuronale Netze sind allgegenwärtig. Immer häufiger werden Sie zum Lösen komplexer Problemstellungen benutzt, in welcher ein regelbasierter Ansatz nicht möglich oder zu komplex erscheint. In den letzten Jahren gab es dabei erhebliche Fortschritte in der Entwicklung, die mittlerweile auch fest in der angekommen sind. ChatGPT ist dabei ein gutes Beispiel was auch aktiv in der breiten Gesellschaft diskutiert wird \parencite{wolfangel_chatgpt_2023}. 

Neuronale Netze zu entwickeln ist dabei ein sehr langwieriger Prozess der oftmals auch sehr viele Ressourcen verbraucht. Dabei spielt die Netztopologie einen wichtigen Faktor, welcher maßgeblich die Performance bestimmt. Eine Optimierung der Hyperparameter ist dabei eine weitere Möglichkeit die Performance des Netzes zu steigern. Zur Optimierung der Parametern werden üblicherweise Methoden wie Grid Search oder die Bayes’sche Optimierung benutzt  \parencite{hutter_hyperparameter_2019}. 

Eine weitere vielversprechende Möglichkeit zur Optimierung der Hyperparameter bietet ein Konzept was sich am Verhalten der Evolution orientiert: die Evolutionären Algorithmen. Es wird eine Population Lösungen erstellt welche dann nach Regeln der Selektion und Mutation eine neue Population bilden \parencite{stelldinger_naturanaloge_2024}

\section{Fragestellung}
Die Hausarbeit beschäftigt sich mit der Optimierung von Hyperparameter eines neuronalen Netzes. Dabei soll mit speziellen Fokus auf die Kodierung und notwendige Operationen zur Rekombination entwickelt werden. Der genetische Algorithmus soll dabei möglichst generalisiert arbeiten, domänenspezifische Rekombination oder Mutationen sind nicht Inhalt dieser Arbeit d.h der Algorithmus soll unabhängig vom verwendeten neuronalen Netz arbeiten. 

