\chapter{Einführung}
\section{Einleitung}\label{sec:einleitung}
Neuronale Netze sind allgegenwärtig. Immer häufiger werden sie zum Lösen komplexer Problemstellungen benutzt, in welcher ein regelbasierter Ansatz nicht möglich oder zu komplex erscheint. In den letzten Jahren gab es dabei erhebliche Fortschritte in der Entwicklung, die mittlerweile auch fest in der Populärkultur angekommen sind. ChatGPT hat in den letzten Jahren einen breiten Diskurs in der Öffentlichkeit ausgelöst. \parencite{wolfangel_chatgpt_2023}. 

Neuronale Netze zu entwickeln ist ein sehr langwieriger Prozess der oftmals auch sehr viele Ressourcen verbraucht. Dabei ist die Netztopologie ein wichtigen Faktor, welcher maßgeblich die Performance bestimmt. Eine Optimierung der Hyperparameter ist dabei eine weitere Möglichkeit die Performance des Netzes zu steigern. Zur Optimierung der Parametern werden üblicherweise Methoden wie Grid Search oder die Bayes’sche Optimierung benutzt  \parencite{hutter_hyperparameter_2019}. 

Eine weitere vielversprechende Möglichkeit zur Optimierung der Hyperparameter bieten die Evolutionären Algorithmen. Es wird eine Population mit Individuen erstellt welche dann nach Regeln der Selektion und Mutation eine neue Population bilden. Dabei werden vielversprechende Lösungen miteinander kombiniert, um neue, bessere Lösungen zu finden. Genetische Algorithmen werden in Optimierungsproblemen verwendet, die über einen großen Suchraum über mehrere Dimensionen verfügen. \parencite{stelldinger_naturanaloge_2024}

\section{Fragestellung}
Meine Hausarbeit beschäftigt sich mit der Optimierung von Hyperparameter eines neuronalen Netzes. Mithilfe eines genetischen Algorithmus soll eine möglichst optimale Konfiguration von Hyperparametern gefunden werden. Dabei soll das unterliegende neuronale Netz als \textit{Blackbox} behandelt werden. Fokus dieser Hausarbeit liegt auf der Optimierung der genetischen Operationen wie der Selektion, Rekombination als auch die Mutation. Zudem soll der implementierte Algorithmus konventionellen Methoden wie Grid Search gegenübergestellt werden.

