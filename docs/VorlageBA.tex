% Dies ist die Hauptdatei, von der aus das Gesamtdokument erzeugt wird.  Diese
% Datei sollten Sie zunächst umbenennen, damit sie keinen generischen Namen
% hat!  Dann wird sie mit LuaLaTeX kompiliert.

% In der Datei defs.tex werden alle globalen LaTeX-spezifischen Einstellungen
% vorgenommen
\input{defs}

% Wenn das Kommentarzeichen entfernt wird, kann man mit einem Befehl wie
%
% \includeonly{chap2,chap3}
%
% erreichen, dass nur ausgewählte Dateien kompiliert werden.  Das ist für die
% Arbeit an umfangreichen Dokumenten hilfreich, weil es Zeit spart.  Für das
% Erstellen des fertigen Dokuments muss der Befehl natürlich wieder
% auskommentiert werden, damit alle Referenzen aktuell sind und die
% Seitenzahlen stimmen.

% Hier beginnt das eigentliche Dokument.  Sie können weitere Dateien
% hinzufügen und natürlich auch vorhandene weglassen.  Die vorhandene
% Dateistruktur ist lediglich als Beispiel gedacht.
\begin{document}
% In dieser Umgebung wird die Titelseite separat vom restlichen Text gesetzt
\begin{titlepage}
  % andere Seitenränder als im Rest der Arbeit
  \newgeometry{lmargin=2cm,tmargin=7mm,rmargin=5mm,bmargin=1cm}
  % die "Hausfarbe" der HAW; diese und die folgenden Einstellungen sind lokal
  % und gelten nur innerhalb der Umgebung "titlepage"
  \color{haw}
  % Blocksatz für die Titelseite deaktivieren
  \raggedright
  % Logo rechtsbündig setzen
  \hfill\includegraphics[width=7cm]{HAW_Marke_RGB_300dpi}\\

  % vertikaler Abstand
  \vspace{5cm}

  % Wahl der "Hausschrift" Open Sans der HAW, die als Schrift auf Ihrem
  % Rechner installiert sein muss
  \setmainfont{Open Sans}
  % etwas kleiner als üblich
  \small
  % fett und in Majuskeln
  \textbf{HAUSARBEIT}

  % vertikaler Abstand
  \vspace{8mm}

  % der Titel der Arbeit als "Seite in der Seite"; natürlich müssen Sie hier
  % Ihren Titel eintragen
  \begin{minipage}{0.8\linewidth}
    % Wahle der zweiten "Hausschrift" der HAW, die ebenfalls auf Ihrem Rechner
    % bereits vorhanden sein muss
    \setmainfont{Martel Heavy}
    % ziemlich große Schrift
    \LARGE
    % [1mm] steht jeweils für einen etwas größeren Durchschuss
    Optimierung der Hyperparameter\\[1mm]
    eines Neuronalen Netzes\\[1mm]
    durch einen\\[1mm]
    evolutionären Algorithmus\\
    % am Ende noch ein waagerechter Strich, das CD will es so...
    \,\rule{11mm}{1.2mm}
  \end{minipage}

  % vertikaler Abstand, überraschenderweise
  \vspace{1cm}

  % hier korrektes Datum und Ihren Namen eingeben
  vorgelegt am 02. August 2024\\
  Jonas Metzger

  % letzter vertikaler Abstand für heute
  \vspace{5cm}

  % noch eine "Seite in der Seite", etwas nach rechts geschoben
  \hspace*{37mm}
  \begin{minipage}{0.5\linewidth}
    % Namen und Titel der beiden Prüfer eintragen
    \begin{tabular}{@{}ll}
      Erstprüferin: & Prof. Dr. Peer Stelldinger\\[-.3mm]
    \end{tabular}\\

    % noch ein horizontaler Strich
    \,\rule{9mm}{1mm}\\[1.5mm]

    \textbf{HOCHSCHULE FÜR ANGEWANDTE}\\
    \textbf{WISSENSCHAFTEN HAMBURG}\\
    Department Informatik\\
    Berliner Tor 7\\
    20099 Hamburg
  \end{minipage}
\end{titlepage}
% setzt die Geometrie wieder auf die Standardwerte zurück
\restoregeometry

% für die Seite mit dem Abstract keine Seitenzahl ausgeben
\thispagestyle{empty}
\include{toc}
\include{chap1}
\include{chap2}
\include{chap3}
% -*- coding: utf-8 -*-

% Ausgabe des Literaturverzeichnisses; ohne weitere Optionen werden nur die
% Bücher und Artikel ausgegeben, die in der Arbeit auch zitiert werden.
\printbibliography

% markiert den Anfang des Anhangs
\appendix

% ein Kapitel, das nicht numeriert, aber trotzdem ins Inhaltsverzeichnis
% aufgenommen wird
\clearpage
\section*{Anhang}
\begin{figure}[p]
	\centering
	\includegraphics[width=1\linewidth]{acc.png}
	\caption{Lauf mit Priorisierung der Exploitation: Höchste Genauigkeit pro Generation}
	\label{fig:enter-label}
\end{figure}

\begin{figure}[p]
	\centering
	\includegraphics[width=1\linewidth]{loss_explore.png}
	\caption{Lauf mit Priorisierung der Exploration: Geringster Fehler pro Generation}
	\label{fig:enter-label}
\end{figure}

\begin{figure}[p]
	\centering
	\includegraphics[width=1\linewidth]{acc.png}
	\caption{Lauf mit Priorisierung der Exploration: Höchste Genauigkeit pro Generation}
	\label{fig:enter-label}
\end{figure}


Der Sourcecode steht auf Github zur Verfügung:\\
\url{https://github.com/jonasmetzger2000/KI-STG-GI-Neural-Network-Optimizer}

% neue Seite
\clearpage

% keine Seitenzahl
\thispagestyle{empty}

% keine Nummerierung, keine Aufnahme ins Inhaltsverzeichnis
\section*{Eigenständigkeitserklärung}

% Hier müssen Sie natürlich den Titel der Arbeit sowie Ort und Datum ersetzen:
Hiermit versichere ich, dass ich die vorliegende Bachelorarbeit mit dem Titel
\begin{center}
  \textbf{Optimierung der Hyperparameter eines Neuronalen Netzes durch einen evolutionären Algorithmus}
\end{center}
selbstständig und nur mit den angegebenen Hilfsmitteln verfasst habe.  Alle
Passagen, die ich wörtlich aus der Literatur oder aus anderen Quellen wie
z.\,B. Internetseiten übernommen habe, habe ich deutlich als Zitat mit Angabe
der Quelle kenntlich gemacht.

\vspace{2cm}

Hamburg, 2. August 2024

\end{document}